\section{Power Analysis}
\noindent
The data distribution goes from 2020-04 to 2022-10, showing a discrete amount of information that can be enough to make some initial observations and guessings. \\
The lines in the plots that have a strange or unusual pattern are the result of the horizontal and vertical interpolation applied to fill some empty spaces in the distribution.

\vspace{-15pt}

\begin{figure}[H]
\centering
\includegraphics[width=1\textwidth]{../../PLOTS/PWR_total.png}
\captionsetup{skip=-10pt}
\caption{PWR total value (sum of all nodes in the server)}
\label{fig:PWR_total}
\end{figure}

\begin{center}
\setstretch{0.9}
count    2.110200e+04 \\
mean     7.997859e+05 \\
std      1.008300e+05 \\
min      3.101649e+04 \\
25\%      7.600741e+05 \\
50\%      8.096636e+05 \\
75\%      8.581640e+05 \\
max      1.311606e+06
\end{center}

In terms of pure power consumption, we see a visible peak that reaches a value of 1.3 MW, while the general mean stays close to 0.8 MW. The data fluctuates a lot throughout the days, but it is difficult to find any particular pattern or repetition at this level of depth; what we can guess is that all o most of the lowest points in the plot are given by a moment or period of maintenance for the server, while the highest values might indicate an episode of testing for the capabilities of the server in terms of maximum computational power.

\subsection{PWR r205}
At rack or even node level we see a similar behaviour: a lot of peaks rising from an horizontal line that indicates the mean power consumption. 

\vspace{-15pt}

\begin{figure}[H]
\centering
\includegraphics[width=1\textwidth]{../../PLOTS/PWR_r205.png}
\captionsetup{skip=-10pt}
\caption{PWR r205}
\label{fig:PWR_r205}
\end{figure}

\vspace{-20pt}

\begin{figure}[H]
\centering
\includegraphics[width=1\textwidth]{../../PLOTS/PWR_r205n01.png}
\captionsetup{skip=-10pt}
\caption{PWR r205n01}
\label{fig:PWR_r205n01}
\end{figure}

\subsection{PWR r206}
The first rack (r205) seems to be the only one with such a regular and stable power consumption; indeed just looking at a different rack like this one (but every other rack is more similar to this) we see a much different and oscillating plot. The only hypothesis we can make is that the first rack is much less used that all the others, or maybe it is used for a different purpose.

\vspace{-12pt}

\begin{figure}[H]
\centering
\includegraphics[width=1\textwidth]{../../PLOTS/PWR_r206.png}
\captionsetup{skip=-10pt}
\caption{PWR r206}
\label{fig:PWR_r206}
\end{figure}

\vspace{-20pt}

\begin{figure}[H]
\centering
\includegraphics[width=1\textwidth]{../../PLOTS/PWR_r206n01.png}
\captionsetup{skip=-10pt}
\caption{PWR r206n01}
\label{fig:PWR_r206n01}
\end{figure}

\subsection{PWR r206n01 STL}
Even making a seasonal-trend decomposition it’s hard to highlight any specific trend, since the data is full of variations and outliers. Taking advantage of a different tool we’ll try to make any seasonality clearer.

\vspace{-10pt}

\begin{figure}[H]
\centering
\includegraphics[width=1\textwidth]{../../PLOTS/PWR_stl_r206n01.png}
\caption{PWR r206n01 STL}
\label{fig:PWR_stl_r206n01}
\end{figure}